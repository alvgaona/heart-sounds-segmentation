\chapter{Presentación} \label{ch:presentation}

\section{Motivación} \label{sec:motivation}

\indent El presente trabajo se hace en el marco de una tesis de grado de la \textit{Facultad de Ingeniería de la
Universidad de Buenos Aires} (FIUBA) en conjunto con los investigadores Dr. Pedro David Arini y la Dra. María Paula
Bonomini. \bigskip

\indent Pedro y Paula se encuentran actualmente en el \textit{Instituto Argentino de Matemática} (\acrshort{iam})
del \textit{Consejo Nacional de Investigaciones Científicas y Técnicas} (\acrshort{conicet}) y en el
\textit{Instituto de Investigaciones Biomédicas ``Alberto Sols``} (\acrshort{iibm}).
Pedro se especializa en el campo de \textit{``Procesamiento digital de señales aplicado al cálculo y modelado
estadístico de la actividad eléctrica cardíaca``} y Paula en el campo de \textit{``Modelos matemáticos y tratamiento
digital de la señal electrocardiográfica para predicción de riesgo cardíaco``}. \bigskip

\indent Por otro lado, el análisis de fonocardiogramas no es común en ambientes clínicos como hospitales.
Dicho esto conseguir datos asociados de pacientes no es una tarea sencilla.
Por ende, el \acrshort{iam} se encuentra trabajando en el diseño e implementación de un equipo capaz de realizar la
adquisición de estas señales con bajo error para realizar una base de datos propia dedicada a la investigación.

\indent Más aún, las técnicas de análisis automático de fonocardiogramas no han sido desarrolladas como el análisis
de electrocardiogramas.
Así, el \acrshort{iam} tiene pendiente como tema de investigación la segmentación de fonocardiogramas para luego que
sea ésta una base en la detección automática de enfermedades cardiovasculares. \bigskip

\indent Finalmente, este tema será continuado en áreas de estudio superiores con el objetivo de mejorar lo presentado y
aportar a la comunidad científica nuevos resultados.

\newpage

\section{Objetivos} \label{sec:objectives}

La propuesta de investigación tiene como enfoque proponer una alternativa de segmentación diferente al estado del
arte. Bajo esta disposición, se definen objetivos generales al tema de investigación y objetivos específicos del
trabajo presentado en este documento. \bigskip

El desempeño de la segmentación es fundamental en todo trabajo relacionado a la detección o estimación. De esta
manera este trabajo ha sido llevado a cabo con ese objetivo.
Por otro lado, se ha visto en la implementación de David Springer \textit{et al.} \cite{pp:springer2015} que el
algoritmo de etiquetamiento comete errores, lo cual requiere un ajuste ad hoc de los parámetros. Ésto dispara la
necesidad de un nuevo algoritmo y/o la extracción de marcas de otras señales complementarias. \\
\indent De la mano del desempeño y del error de segmentación se encuentra la necesidad de ampliar la base de
datos de entrenamiento.

\subsection*{Generales} \label{subsec:general-objectives}

\begin{itemize}
  \item Mejorar el desempeño de la segmentación respecto al estado del arte.
  \item Proponer un algoritmo de etiquetamiento automático a partir de marcas del \acrshort{ecg}.
  \item Proponer nuevas marcas asociadas a otras señales complementarias al \acrshort{pcg} (\textit{phonocardiogram}).
  \item Ampliar la base de datos utilizada para la segmentación.
  \item Realizar una implementación en tiempo real del segmentador propuesto.
  \item Proponer un clasificador para la detección de señales patológicas.
  \item Plantear nuevas líneas de trabajo que permitan continuar con el trabajo de investigación.
\end{itemize}

\subsection*{Específicos} \label{subsec:specific-objectives}

\begin{itemize}
  \item Extraer marcas de \acrshort{ecg} asociados a un \acrshort{pcg}.
  \item Acondicionar las señales de \acrshort{pcg}.
  \item Implementar algoritmo de extracción de cuadros (\textit{frames}) de las señales de \acrshort{pcg}.
  \item Implementar red neuronal \acrshort{lstm} como clasificador.
  \item Implementar un entrenamiento del modelo de segmentación bajo el concepto de \textit{Cross-validation}.
  \item Comparar métricas de desempeño entre los distintos métodos del estado del arte.
  \item Proponer mejoras a futuro.
\end{itemize}

\newpage

\section{Lineamientos generales de la tesis} \label{sec:general-topics}

\begin{itemize}
  \item \textbf{Capítulo 1}: Se hace una presentación del trabajo. Explica el contexto, motivación y objetivos.
  \item \textbf{Capítulo 2}: Se da un marco teórico de la naturaleza de la señal y de la fisiología asociada al tema,
  dando un contexto de la biología humana involucrada.
  \item \textbf{Capítulo 3}: Se hace mención a las distintas bases de datos con fonocardiogramas, la comparativa entre
  ellas.
  \item \textbf{Capítulo 4}: Se explica la etapa de pre-procesamiento necesaria para el acondicionamiento de las
  señales (filtros, etiquetas, ente otras cuestiones).
  \item \textbf{Capítulo 5}: Se explica la etapa de procesamiento del trabajo, abordando principalmente la teoría
  de la extracción de atributos de los fonocardiogramas.
  \item \textbf{Capítulo 6}: Se aborda el tema de \textit{deep learning} definiendo el marco teórico de los modelos
  y técnicas aplicados en este trabajo.
  \item \textbf{Capítulo 7}: Se muestra la implementación de la solución al problema en cuestión, mostrando algoritmos
  implementados, arquitectura del modelo.
  \item \textbf{Capítulo 8}: Se intenta ilustrar los resultados del trabajo y proponiendo algunos temas de discusión
  a partir de estos resultados obtenidos.
  \item \textbf{Capítulo 9}: Se toman conclusiones del trabajo, proponiendo mejoras a fin de igualar o superar el
  rendimiento de otros algoritmos del actual estado del arte.
\end{itemize}
