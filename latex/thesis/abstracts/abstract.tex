\begin{abstract}
This thesis presents the analysis of a phonocardiogram segmenter based on the set of techniques
known as \textit{Deep Learning}.
The segmentation of these signals, that can be pathological and non-pathological, are analyzed using a
\textit{Recurrent Neural Network} (\acrshort{rnn}).
Particularly this neural network is called \textit {Long Short Term Memory} (\acrshort {lstm}).
This application based on an LSTM network, with performance metrics close to that of the state of the art can be
carried out for the implementation of a real-time segmenter.
To fully understand the operation of the segmenter, basic concepts will be developed about the modeling of a LSTM unit
and the different layers applied for shaping the segmenter, followed by a brief introduction of pre-processing,
feature extraction and a potential post-processing instance.
The concept of recurrent neural networks is the nucleus of this thesis and it is explained how from this neural network
it is possible to segment these signals, which are relatively amorphous to other known signals.
Among these other signals, the electrocardiogram (\acrshort {ecg}) is one of them.
The algorithm needs prior information from a balanced data set between pathological and non-pathological signals for
training.
Additionally, the delineation of the \acrshort{ecg} to extract the R wave and the end of the T wave is an important
task for this thesis.
The implementation is based on \textsc{Matlab \texttrademark} and calculates the performance of the system
based on standardized metrics in the classification literature.
\end{abstract}