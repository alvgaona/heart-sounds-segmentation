\begin{center}
\section*{\Huge{RESUMEN}}
\end{center}

Esta tesis presenta el analisis de  un segmentador de señales de fonocardiograma basado en el conjunto de técnicas
conocidas como \textit{Deep Learning}. Se analiza la segmentación de estas señales que pueden ser de naturaleza
patológica y no patológicas, mediante una \acrshort{rnn} (\textit{Recurrent Neural Network}). Particularmente esta
red neuronal se denomina \textit{Long Short Term Memory} (\acrshort{lstm}). Esta aplicación basada en una red LSTM
con métricas de performance cercanas a la del estado del arte pueden ser llevadas a cabo para la implementación de
un segmentador en tiempo real. Para entender el funcionamiento del segmentador, se desarrollarán conceptos básicos
acerca del modelado de la unidad básica de la red LSTM y las distintas capas aplicadas para la conformación del
segmentador, seguido por una breve introducción del pre-procesamiento, extracción de features y de una potencial
instancia de post-procesamiento. El concepto de redes neuronales recurrentes (\textit{recurrent neural network}) es
el núcleo de esta tesis y se explica cómo a partir de esta red neuronal es posible segmentar dichas señales que son
relativamente amorfas a otras señales conocidas. Entre estas se encuentra, el electrocariograma (\acrshort{ecg}),
donde sus ondas están bien definidas. El algoritmo necesita información previa de un set de datos equilibrado entre
señales patológicas y no patológicas para el entrenamiento. Complementariamente, es vital la delineación de los
\acrshort{ecg} extrayendo de esto la onda R y el fin de la onda T. La implementación se encuentra basada en
\textsc{Matlab\texttrademark} y se calcula el desempeño (\textit{performance}) del sistema, basado en métricas
estándar en la literatura de la clasificación y/o segmentación.
