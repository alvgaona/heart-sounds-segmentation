\begin{thebibliography}{9}
  \bibitem{pp:keith_flack}
  A. Keith and M. Flack. \textit{"The form and nature of the muscular connections between the primary divisions of the
  vertebrate"}. Apr, 4. 1907.

  \bibitem{pp:liang}
  H. Liang, S. Lukkarinen, and I. Hartimo, \textit{“Heart Sound Segmentation Algorithm based on heart sound
  Envelogram”}, in Computers in Cardiology, vol. 24, Lund, Swed, 1997, pp. 105–108.

  \bibitem{pp:liang2}
  H. Liang, L. Sakari, and H. Iiro, \textit{“A Heart Sound Segmentation algorithm using Wavelet Decomposition and
  Reconstruction”} in Proceedings of the 19th Annual International Conference of the IEEE Engineering in Medicine and
  Biology Society, vol. 4, Chicago, IL, USA, 1997, pp. 1630–1633.

  \bibitem{bk:boron3ed}
  Boron, W. F. and Boulpaep E. L. \textit{Fisiología Médica}, 3.$^a$ ed. 2017 Elsevier España, S.L.U.

  \bibitem{pp:hochreiter-schmidhuber}
  S. Hochreiter and J. Schmidhuber. \textit{"Long short-term memory. Neural Computation"}, 9(8):1735–1780, 1997.

  \bibitem{pp:martinez2004}
  J. P. Martínez, R. Almeida, S. Olmos, A. P. Rocha, P. Laguna. \textit{"Wavelet-Based ECG Delineator: Evaluation on
  Standard Databases"}. Apr, 2. 2004

  \bibitem{pp:schmidt2010}
  S. E Schmidt, C. Holst-Hansen, C. Graff, E. Toft and JJ. Struijk. \textit{"Segmentation of Heart Sound recordings by
  a Duration-dependent Hidden Markov model"}. Jan, 22. 2010

  \bibitem{pp:zhang2011}
  X. J.  Hu,  J. W Zhang  G. T Cao,  H.H Zhu ;  H. Li. \textit{"Feature Extraction and Choice in PCG based on Hilbert
  Transfer"}. Dec, 12. 2011.

  \bibitem{pp:abbas2014}
  A. Abbas, R. Bassam and R. Mazin. \textit{"Automated Pattern Classification for PCG Signal based on Adaptive
  Spectral K-means Clustering Algorithm"}. Mar, 15. 2014

  \bibitem{pp:springer2015}
  D. Springer, L. Tarassenki and G. Clifford. \textit{Logistic Regression-HSMM-based Heart Sound Segmentation}. Sep, 1
  . 2015

  \bibitem{pp:renna2018}
  F. Renna and M. Coimbra. \textit{Deep Convolution Neural Netwok for Heart Sound Segmentation}. Jan, 24. 2019

  \bibitem{ref:logi-regression-springer}
  D. Springer,\textit{"Logistic Regression-HSMM-based Heart Sound Example Code"}, Physionet. 2016.


  \bibitem{ref:illanes-zhang}
  A. Illanes-Manriquez and Q. Zhang, “An algorithm for QRS onset and
  offset detection in single lead electrocardiogram records,” in 29th Annual International Conference of the IEEE
  Engineering in Medicine and Biology Society, Lyon, France, 2007, pp. 541 – 544.

  \bibitem{ref:behar}
  J. Behar, et al., \textit{“A comparison of single channel fetal ecg extraction methods”}, Annals of Biomedical
  Engineering, vol. 42, no. 6, pp. 1340–
  1353, June 2014.

  \bibitem{ref:behar-oster-clifford}
  J. Behar, J. Oster, and G. D. Clifford, \textit{“Combining and Benchmarking
  Methods of Foetal ECG Extraction Without Maternal or Scalp Electrode Data”}, Physioogical Measurement, vol. 35, no.
  8, pp. 1569–1589, Aug.
  2014.

  \bibitem{ref:zhang}
  Q. Zhang, et al., \textit{“An algorithm for robust and efficient location of T wave ends in electrocardiograms”}.
  IEEE Transactions on Biomedical Engineering, vol. 53, no. 12 (Pt 1), pp. 2544–52, Dec. 2006.

  \bibitem{ref:seisdedos}
  C. R. Vázquez-Seisdedos, et al., \textit{“New approach for T-wave end
  detection on electrocardiogram: performance in noisy conditions”}.
  Biomedical Engineering Online, vol. 10, no. 1, p. 77, Jan. 2011

  \bibitem{ref:messer}
  S. R. Messer, J. Agzarian, and D. Abbott, “Optimal wavelet denoising
  for phonocardiograms,” Microelectronics Journal, vol. 32, no. 12, pp.
  931–941, Dec. 2001.

  \bibitem{ref:kumar}
  D. Kumar, et al., “Noise detection during heart sound recording using
  periodicity signatures.” Physiological Measurement, vol. 32, no. 5, pp. 599–618, May 2011.

  \bibitem{ref:oskiper-watrous}
  T. Oskiper and R. Watrous, “Detection of the first heart sound using a time-delay neural network,” in Computers in
  Cardiology. Memphis,
  TN, USA: IEEE, 2002, pp. 537–540.

  \bibitem{ref:ergen-tatar-gulcur}
  B. Ergen, Y. Tatar, and H. O. H. Gulcur, “Time-frequency analysis of
  phonocardiogram signals using wavelet transform: a comparative study,” Computer Methods in Biomechanics and
  Biomedical Engineering, no. October, pp. 37–41, Jan. 2011

  \bibitem{ref:liang-sakari-iiro}
  H. Liang, L. Sakari, and H. Iiro, “A heart sound segmentation algorithm using wavelet decomposition and
  reconstruction,” in Proceedings of the 19th Annual International Conference of the IEEE Engineering in Medicine and
  Biology Society, vol. 4, Chicago, IL, USA, 1997, pp. 1630–1633.

  \bibitem{ref:gupta}
  C. Gupta, et al., “Neural network classification of homomorphic
  segmented heart sounds,” Applied Soft Computing, vol. 7, no. 1, pp.
  286–297, Jan. 2007.

  \bibitem{ref:daubechies-maes}
  I. Daubechies and S. Maes, “A nonlinear squeezing of the continuous wavelet transform based on auditory nerve
  models,” Wavelets in Medicine and Biology, pp. 527–546, 1996.

  \bibitem{ref:gabor}
  D. Gabor, “Theory of communication. part 1: The analysis of information,” Journal of I.E.E., vol. 93, no. 26, pp.
  429–441, 1946.

  \bibitem{ref:grossmann-morlet}
  A. Grossmann and J. Morlet, “Decomposition of Hardy functions into square integrable wavelets of constant shape,”
  SIAM journal on mathematical analysis, vol. 15, no. 4, pp. 723–736, 1984.

  \bibitem{ref:gill-gavrieli-intrator}
  D. Gill, N. Gavrieli, and N. Intrator, “Detection and identification of heart sounds using homomorphic envelogram
  and self-organizing probabilistic model,” in Computers in Cardiology, Lyon, France, 2005, pp. 957–960.

  \bibitem{pp:adam}
  Kingma, Diederik, and Jimmy Ba. \textit{\" Adam: A method for stochastic optimization."}, 3rd International
  Conference for Learning Representations, San Diego, 2015.

  \bibitem{pp:oliveira-renna-coimbra}
  J. Oliveira, F. Renna, and M. T. Coimbra, “Adaptive sojourn time HSMM for heart sound segmentation,” IEEE J. Biomed.
  Health Informatics, 2018, early access.

  \bibitem{pp:alexnet}
  A. Krizhevsky, I. Sutskever, G. E. Hinton, \textit{``ImageNet Classification with Deep Convolutional
  Neural Networks``}. ImageNet LSVRC-2010.

  \bibitem{pp:googlenet}
  C. Szegedy, W. Liu, Y. Jia, P. Sermanet, S. Reed, D. Anguelov, D. Erhan, V. Vanhoucke, A. Rabinovich,
  \textit{``Going Deeper with Convolutions``}. CVPR 2015.
\end{thebibliography}
